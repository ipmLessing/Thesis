\newglossaryentry{PhD}{name={PhD}, description={Doctor of Philosophy}}
\newglossaryentry{R}{name={$\mathbb{R}$}, description={Real Number}}
\newglossaryentry{FDCT}{name={FDCT}, description={Science and Technology Development Fund}}

\chapter{Introduction}\label{ch:1}
\minitoc

The main objective of this template aims at providing guidelines and procedures for \gls{PhD} students to undertake their PhD Confirmatory Examination, which is one of the major requirements for the students to progress to the Confirmation of Candidature. This is considered as the specifications prepared by the PhD in Computer Applied Technology Program in the School of Applied Sciences.

\newpage

\section{General overview of the field of study}

Discuss the importance of adaptive control in autonomous systems, particularly the need to balance performance (e.g., speed, maneuverability) with sustainability (e.g., battery health or SOH).

Highlight the relevance of integrating perception (e.g., drone racing) and SOH management (e.g., EV batteries) into control systems.



\section{Research Objectives}

Propose a reinforcement learning (RL)-based control framework for balancing performance and SOH preservation.

Outline how perception (YOLOv8) and SOH predictions (Bi-LSTM) are key components in the architecture.

Emphasize the conceptual design and case studies as proof-of-concept contributions.


\subsection{Contributions of the Thesis}

Conceptual design of an SOH-aware RL controller integrating perception and energy management.

Case study: YOLOv8-based drone racing perception using synthetic and real-world data.

Case study: Bi-LSTM for battery SOH prediction and energy-aware decision-making.

Framework for simulation-to-reality transfer using Blender and Grounding DINO.


\subsubsection{Achievement}

\begin{itemize}
	\item The award of ``Scientific and Technological R\&D Award for Postgraduates'' by the Science and Technology Development Fund (\gls{FDCT}) in November 2022.
\end{itemize}

\subsubsection{Publication}

\begin{itemize}\sloppy\emergencystretch=1em
	\item \bibentry{Picasso1999-Les}
	\item \bibentry{Picasso1995-Le}
\end{itemize}

More information about Picasso can be found on the Wikipedia~\cite{picasso1881wiki}, or in the book: ``Pablo Picasso: The Man and the Image''~\cite{lyttle1989pablo}.


\section{Thesis Structure}

Provide a roadmap summarizing the subsequent chapters and their contributions.

\endinput