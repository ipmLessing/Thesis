\newglossaryentry{EV}{name={EV}, description={Electric Vehicle}}
\newglossaryentry{Bi-LSTM}{name={Bi-LSTM}, description={Bidirectional Long Short-Term Memory}}
\newglossaryentry{OBD-II}{name={OBD-II}, description={On-Board Diagnostic II}}
\newglossaryentry{SOC}{name={SOC}, description={State of Charge}}
\newglossaryentry{BMS}{name={BMS}, description={Battery Management System}}
\newglossaryentry{RL}{name={RL}, description={Reinforcement Learning}}
\newglossaryentry{SOH}{name={SOH}, description={State of Health}}
\newglossaryentry{MAE}{name={MAE}, description={Mean Absolute Error}}
\newglossaryentry{CNN}{name={CNN}, description={Convolutional Neural Network}}
\newglossaryentry{MLP}{name={MLP}, description={Multilayer Perceptron}}


\chapter{Case Studies}\label{ch:5}
\minitoc

Schedules and Milestones

\newpage

\section{Perception Case Study: Drone Racing}

Present results from YOLOv8-based perception:

Compare accuracy between real-world and synthetic data (e.g., 75\% vs. 65\%).

Highlight the impact of Blender and Grounding DINO on model performance.


\section{Driving Behavior and Thermal Dynamics: A Case Study of EV Batteries}
\label{sec:thermal_dynamics_case_study}

Understanding the relationship between driving behavior and battery thermal dynamics is critical for improving the performance and longevity of \glspl{EV}. 
Thermal conditions directly impact battery efficiency, safety, and long-term degradation. 
This case study explores how different driving behaviors influence battery temperature using a large-scale dataset collected from a Nissan Leaf \gls{EV}. 
A \gls{Bi-LSTM} model is employed to predict battery temperature, providing insights into thermal management strategies and their potential role in preserving battery health.

\subsection{Dataset and Data Collection}
The data for this study was collected over a year using a custom IoT device connected to the \gls{EV}'s \gls{OBD-II} port. The dataset comprises over 4.5 million samples, recorded at a temporal resolution of 250 milliseconds, spanning 7,100 kilometers of real-world driving. Key recorded parameters include:

\begin{itemize}
    \item \textbf{Battery Metrics:} Voltage, current, pack temperature, and \gls{SOC}.
    \item \textbf{Driving Behavior:} Acceleration, braking pedal pressure, and speed.
    \item \textbf{Environmental Factors:} Ambient temperature, weather conditions, and seasonal variations.
\end{itemize}

The dataset captures diverse driving styles, environmental conditions, and road types, enabling a detailed analysis of how external and behavioral factors influence battery temperature.

% \begin{figure}[ht]
%     \centering
%     \includegraphics[width=\textwidth]{figures/seasonal_variation.png} % Replace with the path to Figure 4
%     \caption{Seasonal variations of battery pack temperature.}
%     \label{fig:seasonal_variation}
% \end{figure}

% \begin{figure}[ht]
%     \centering
%     \includegraphics[width=\textwidth]{figures/braking_pressure.png} % Replace with the path to Figure 5
%     \caption{Histogram of braking pressure and its correlation with aggressive driving behavior.}
%     \label{fig:braking_pressure}
% \end{figure}

\subsection{Driving Behavior and Battery Thermal Conditions}
Driving behavior significantly impacts battery temperature. This section categorizes driving styles and examines their thermal effects:

\begin{itemize}
    \item \textbf{Gentle Driving:} Smooth accelerations and decelerations kept battery temperatures within the optimal range (15°C–35°C), minimizing stress on the battery and maintaining efficiency.
    \item \textbf{Aggressive Driving:} Sharp accelerations and frequent braking led to temperature surges exceeding 41°C, particularly in high ambient temperatures, imposing additional thermal stress on the battery.
    \item \textbf{Cautious Driving:} Minimal pedal input was typical, but during winter conditions, prolonged exposure to suboptimal temperatures (<15°C) resulted in efficiency losses and potential risks to long-term performance.
\end{itemize}

% \begin{figure}[ht]
%     \centering
%     \includegraphics[width=\textwidth]{figures/temp_aggressive.png} % Replace with the path to Figure 7
%     \caption{Temperature distribution under aggressive driving.}
%     \label{fig:temp_aggressive}
% \end{figure}

% \begin{figure}[ht]
%     \centering
%     \includegraphics[width=\textwidth]{figures/temp_winter.png} % Replace with the path to Figure 8
%     \caption{Temperature trends during winter driving.}
%     \label{fig:temp_winter}
% \end{figure}

\subsection{Bidirectional Long Short-Term Memory Model for Temperature Prediction}
To predict battery pack temperature under varying conditions, a \gls{Bi-LSTM} model was developed. The architecture and training process are described below:

\subsubsection{Model Architecture}
The \gls{Bi-LSTM} network consists of three stacked LSTM layers followed by dense layers for regression. It incorporates features such as:
\begin{itemize}
    \item Driving behavior metrics (acceleration, braking pressure).
    \item Battery parameters (current, \gls{SOC}).
    \item Environmental factors (ambient temperature).
\end{itemize}

\subsubsection{Feature Engineering}
Time-series features were engineered to capture temporal dependencies:
\begin{itemize}
    \item Metrics such as cumulative acceleration and braking frequency over 10-minute windows.
    \item Temporal encoding of cyclic features, such as time of day and seasonal effects.
\end{itemize}

\subsubsection{Training Process}
Data was segmented into overlapping 10-minute windows. Model hyperparameters were optimized using Optuna, achieving:
\begin{itemize}
    \item A \gls{MAE} of 2.92°C on the test set.
    \item An \gls{MAE} of 1.7°C during cross-validation.
\end{itemize}
The \gls{Bi-LSTM} outperformed baseline models, including \glspl{CNN} and \glspl{MLP}, across key performance metrics.

% \begin{figure}[ht]
%     \centering
%     \includegraphics[width=\textwidth]{figures/bilstm_architecture.png} % Replace with the path to Figure 10
%     \caption{Bi-LSTM architecture with feature flow.}
%     \label{fig:bilstm_architecture}
% \end{figure}

\begin{table}[ht]
    \centering
    \caption{Performance comparison of \gls{Bi-LSTM} with baseline models.}
    \label{tab:model_performance}
    \begin{tabular}{|c|c|c|c|}
        \hline
        Model & MAE (°C) & RMSE (°C) & Accuracy (\%) \\ \hline
        \gls{Bi-LSTM} & 2.31 & 2.92 & 95.6 \\ \hline
        \gls{CNN} & 3.15 & 3.89 & 90.2 \\ \hline
        MLP & 3.42 & 4.01 & 88.7 \\ \hline
    \end{tabular}
\end{table}

\subsection{Results}
The \gls{Bi-LSTM} model demonstrated robust performance in capturing the effects of driving behavior on battery temperature:

\subsubsection{Prediction Accuracy}
The model effectively predicted:
\begin{itemize}
    \item Temperature surges caused by aggressive driving.
    \item Low-temperature conditions during winter driving.
\end{itemize}

\subsubsection{Behavioral Analysis}
Aggressive driving exhibited occasional overestimations due to rapid transitions, while gentle and cautious driving resulted in stable predictions.

% \begin{figure}[ht]
%     \centering
%     \includegraphics[width=\textwidth]{figures/prediction_vs_actual.png} % Replace with the path to Figure 12
%     \caption{Prediction vs. actual temperature for different driving styles.}
%     \label{fig:prediction_vs_actual}
% \end{figure}

% \begin{figure}[ht]
%     \centering
%     \includegraphics[width=\textwidth]{figures/error_distribution.png} % Replace with the path to Figure 13
%     \caption{Error distribution across seasons and driving behaviors.}
%     \label{fig:error_distribution}
% \end{figure}

\subsection{Discussion}
\textbf{Key Insights:}
\begin{itemize}
    \item Aggressive driving imposes significant thermal stress on the battery, increasing the risk of accelerated degradation.
    \item Gentle driving minimizes temperature fluctuations, preserving thermal stability and potentially extending battery lifespan.
\end{itemize}

\textbf{Implications for \gls{BMS}:}
\begin{itemize}
    \item The Bi-LSTM model can enable real-time thermal management by predicting temperature trends and issuing proactive alerts for high-risk behaviors.
    \item Integrating temperature predictions into \gls{BMS} could optimize battery health by dynamically adjusting operating conditions.
\end{itemize}

\textbf{Limitations and Future Work:}
\begin{itemize}
    \item The model's accuracy depends on high-quality data, and edge cases with rapid behavioral shifts remain challenging.
    \item Future research should explore how thermal predictions can directly influence \gls{SOH} management strategies.
\end{itemize}


\section{Proof-of-Concept Simulation for the \gls{RL} Framework}

Simulate a simplified scenario where perception and SOH predictions are used to inform high-level decision-making.

Use existing datasets to demonstrate how the \gls{RL} architecture could function in practice, focusing on:

Mode-switching behavior.

Trade-offs between performance and \gls{SOH} metrics.

\endinput