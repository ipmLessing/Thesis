\chapter{Literature Review}\label{ch:2}
\minitoc

The literature review builds a knowledge base and a set of potential paths for the upcoming technical details to solve the problems.
It usually serves as the illustration of foundation, evidence, state-of-the-art, and critiques.
Through citing other established pieces of work, it is here to convince yourself and the readers of your work how and why you narrow the ideas down to the formally formulated problems,
and also to help understand why you choose the particular research paths and approaches in your methodology.

\newpage


\section{Perception in High-Speed Autonomous Systems}

Review state-of-the-art YOLO models and their applications in navigation and drone racing.

Discuss your work on Drone Racing Perception with YOLOv8, keypoint detection, and PNP-based reprojection.


\section{Battery SOH Prediction for Autonomous Systems}

Summarize SOH management techniques, including Bi-LSTM models for battery health prediction.

Include findings from your ICECI and CCNC papers.


\section{Reinforcement Learning in Multi-Objective Control}

Review RL applications in control systems, with a focus on balancing multiple objectives (e.g., performance and energy efficiency).

Discuss the challenges of integrating RL with real-world constraints like SOH predictions and perception inputs.


\section{Generalization and Synthetic Data for Autonomous Systems}

Explore strategies like domain randomization, synthetic data generation, and few-shot learning for improving model robustness.


Highlight your work on Blender-generated datasets and Grounding DINO for efficient training with minimal data.


\section{Ethical, Societal, and Regulatory Considerations}


Address ethical issues, such as data privacy in perception systems and safety implications of SOH-aware control.

Discuss relevant regulations for deploying autonomous systems in public domains.


\endinput