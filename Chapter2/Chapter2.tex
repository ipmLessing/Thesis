
\newglossaryentry{EV}{name={EV}, description={Electric Vehicle}}
\newglossaryentry{BEV}{name={BEV}, description={Battery Electric Vehicle}}
\newglossaryentry{LiB}{name={LiB}, description={Lithium-Ion Battery}}
\newglossaryentry{NEV}{name={NEV}, description={New Electric Vehicle}}
\newglossaryentry{SOH}{name={SOH}, description={State-Of-Health}}
\newglossaryentry{SOC}{name={SOC}, description={State-Of-Charge}}
\newglossaryentry{DOD}{name={DOD}, description={Depth Of Discharge}}
\newglossaryentry{ECU}{name={ECU}, description={Electronic Control Unit}}
\newglossaryentry{IMU}{name={IMU}, description={Inertial Measurement Unit}}
\newglossaryentry{OBDII}{name={OBD-II}, description={On-Board Diagnostics II}}
\newglossaryentry{IoT}{name={IoT}, description={Internet Of Things}}
\newglossaryentry{ML}{name={ML}, description={Machine Learning}}
\newglossaryentry{CAN}{name={CAN}, description={Controller Area Network}}
\newglossaryentry{ICE}{name={ICE}, description={Internal Combustion Engine}}
\newglossaryentry{CO2}{name={CO\textsubscript{2}}, description={Carbon Dioxide}}
\newglossaryentry{Li-ion}{name={Li-ion}, description={Lithium-Ion}}
\newglossaryentry{GPS}{name={GPS}, description={Global Positioning System}}
\newglossaryentry{AI}{name={AI}, description={Artificial Intelligence}}
\newglossaryentry{NMC}{name={NMC}, description={Lithium Nickel Manganese Cobalt Oxide (LiNiMnGoO\textsubscript{2})}}
\newglossaryentry{SEI}{name={SEI}, description={Solid Electrolyte Interphase}}
\newglossaryentry{WLTC}{name={WLTC}, description={Worldwide Harmonized Light Vehicles Test Cycles}}
\newglossaryentry{TMD}{name={TMD}, description={Transition Metal Dissolution}}
\newglossaryentry{CEI}{name={CEI}, description={Cathode-Electrolyte-Interface}}
\newglossaryentry{BMS}{name={BMS}, description={Battery Management System}}
\newglossaryentry{ANN}{name={ANN}, description={Artificial Neural Network}}
\newglossaryentry{SVM}{name={SVM}, description={Support Vector Machine}}
\newglossaryentry{LiPF6}{name={LiPF\textsubscript{6}}, description={Lithium Hexafluorophosphate}}
\newglossaryentry{EMC}{name={EMC}, description={Ethyl Methyl Carbonate}}
\newglossaryentry{DMC}{name={DMC}, description={Dimethyl Carbonate}}
\newglossaryentry{DEC}{name={DEC}, description={Diethyl Carbonate}}
\newglossaryentry{LiF}{name={LiF}, description={Lithium Fluoride}}
\newglossaryentry{PF5}{name={PF\textsubscript{5}}, description={Phosphorus Pentafluoride}}
\newglossaryentry{HF}{name={HF}, description={Hydrogen Fluoride}}
\newglossaryentry{KF}{name={KF}, description={Kalman Filter}}
\newglossaryentry{SOE}{name={SOE}, description={State-Of-Energy}}
\newglossaryentry{SOP}{name={SOP}, description={State-Of-Power}}
\newglossaryentry{HEV}{name={HEV}, description={Hybrid Electric Vehicle}}
\newglossaryentry{PHEV}{name={PHEV}, description={Plug-In Hybrid Electric Vehicle}}
\newglossaryentry{EOL}{name={EOL}, description={End Of Life}}
\newglossaryentry{RUL}{name={RUL}, description={Remaining Useful Life}}
\newglossaryentry{NEDC}{name={NEDC}, description={EURO 6 New European Driving Cycle}}
\newglossaryentry{HMM}{name={HMM}, description={Hidden Markov Model}}
\newglossaryentry{HHMM}{name={HHMM}, description={Hierarchical Hidden Markov Model}}
\newglossaryentry{GMM}{name={GMM}, description={Gaussian Mixture Model}}
\newglossaryentry{RF}{name={RF}, description={Random Forest}}
\newglossaryentry{DNN}{name={DNN}, description={Deep Neural Network}}
\newglossaryentry{NB}{name={NB}, description={Naive Bayes}}
\newglossaryentry{FNN}{name={FNN}, description={Fuzzy Neural Network}}
\newglossaryentry{KNN}{name={KNN}, description={K-Nearest Neighbor}}
\newglossaryentry{CNN}{name={CNN}, description={Convolutional Neural Network}}
\newglossaryentry{RNN}{name={RNN}, description={Recurrent Neural Network}}
\newglossaryentry{MIC}{name={MIC}, description={Maximal Information Coefficient}}
\newglossaryentry{HSS}{name={HSS}, description={Hybrid-State System Framework}}
\newglossaryentry{S3VM}{name={S3VM}, description={Semi-Supervised Support Vector Machines}}
\newglossaryentry{FL}{name={FL}, description={Fuzzy Logic}}
\newglossaryentry{LSTM}{name={LSTM}, description={Long Short-Term Memory}}
\newglossaryentry{GRU}{name={GRU}, description={Gated Recurrent Unit}}
\newglossaryentry{MLP}{name={MLP}, description={Multilayer Perceptron}}
\newglossaryentry{RPM}{name={RPM}, description={Revolutions Per Minute}}
\newglossaryentry{FC}{name={FC}, description={Fully Connected}}
\newglossaryentry{MFLOPS}{name={MFLOPS}, description={Million Floating-Point Operations Per Second}}
\newglossaryentry{SPMD}{name={SPMD}, description={Safety Pilot Model Deployment}}
\newglossaryentry{BiGRU}{name={BiGRU}, description={Bidirectional Gated Recurrent Unit}}
\newglossaryentry{BiLSTM}{name={BiLSTM}, description={Bidirectional Long Short-Term Memory}}
\newglossaryentry{ReLu}{name={ReLu}, description={Rectified Linear Unit}}
\newglossaryentry{HV}{name={HV}, description={High Voltage}}
\newglossaryentry{CC-CV}{name={CC-CV}, description={Constant Current Constant Voltage}}
\newglossaryentry{AR-HMM}{name={AR-HMM}, description={Auto-Regressive Hidden Markov Model}}


\chapter{Literature Review}\label{ch:2}
\minitoc

The literature review builds a knowledge base and a set of potential paths for the upcoming technical details to solve the problems.
It usually serves as the illustration of foundation, evidence, state-of-the-art, and critiques.
Through citing other established pieces of work, it is here to convince yourself and the readers of your work how and why you narrow the ideas down to the formally formulated problems,
and also to help understand why you choose the particular research paths and approaches in your methodology.

\newpage


\section{Battery Structure and Degradation Mechanisms}

This interdisciplinary study integrates automotive technology, computer applications, and electrochemistry, focusing on key areas such as \gls{LiB}, \gls{IoT}, data analytics, and \gls{AI}. 
The aim is to analyze interconnected factors influencing battery lifespan. 
Among the three types of EVs—\gls{BEV}, \gls{HEV}, and \gls{PHEV}—this research concentrates on \gls{BEV}, the most prevalent EV type, with typical capacities ranging from 20 to 80 kWh.

This study focuses on \gls{LiB} used in \gls{BEV}, emphasizing its structure and degradation. 
Figure~\ref{fig:battery_structure} illustrates a simplified \gls{LiB} configuration. 
As shown by Han et al.~\cite{han2019review}, the battery comprises an anode and cathode intercalated by current collectors (e.g., copper and aluminum) and separated by an insulating sheet. 
The electrolyte ensures material conductivity within the battery shell.

\begin{figure}[hbt]
    \centering
    \includegraphics[width=\textwidth]{Chapter2/figures/bat_fig.pdf}
    \caption{Simplified modern \gls{LiB} structure.}
    \label{fig:battery_structure}
\end{figure}

Various \glspl{LiB} are used in \gls{NEV}, with differing compositions affecting their performance. 
This study highlights batteries with cathode and anode materials of \gls{NMC} ($Li[Ni_xMn_yCo_z]O2$) and graphite, respectively~\cite{Lu2013}. 
While these materials offer high energy density and thermal stability, their poor electrochemical stability leads to chemical reactions during charging and discharging, contributing to degradation~\cite{xu2018progress, han2019review}.

\begin{figure}[hbt]
    \centering
    \includegraphics[width=\textwidth]{Chapter2/figures/overall_aging.pdf}
    \caption{Overview of degradation reactions in \gls{LiB}.}
    \label{fig:aging_reactions}
\end{figure}

Degradation is categorized into capacity fading, measurable via \gls{SOH}, and power fading, determined by internal resistance changes using $P = UI = U^2/R$~\cite{zhang2018state}. The subsequent subsections detail specific degradation mechanisms.

\subsubsection{Anode}

Graphite, the dominant anode material, is affected primarily by \gls{SEI} film formation~\cite{Zhao2017}. 
During the initial charge, 5–10\% of \glspl{Li-ion} form a protective layer, enabling \glspl{Li-ion} permeability but making the layer prone to cracking~\cite{an2016state}. 
These cracks, exacerbated under high \gls{DOD} and \gls{SOC}, interact with the electrolyte, consuming \glspl{Li-ion} and generating gas~\cite{balakrishnan2006safety, guo2021lithium}. 
This process thickens the \gls{SEI} layer, leading to capacity loss.

\subsubsection{Cathode}

The \gls{NMC} cathode, introduced in the 1990s, has undergone extensive research to identify degradation mechanisms~\cite{ma13081884}. 
Table~\ref{tab:cathode_aging} summarizes these processes. 
Of particular importance is the \gls{CEI} film, which forms due to \glspl{Li-ion} loss via electrolyte interactions, similar to the \gls{SEI} film~\cite{jie2020advanced}.

\begin{table}[!ht]
    \centering
    \caption{Key degradation mechanisms in \gls{LiB} cathodes.}
    \begin{tabularx}{\textwidth}{|X|X|X|}
        \hline
        Process & Cause & Consequence \\ \hline
        Phase transition~\cite{VETTER2005269} & High current rates & Structural disorder in cathode \\ \hline
        Particle cracking~\cite{yan2017intragranular} & Prolonged high voltage/current & Impaired \glspl{Li-ion} diffusion, cathode collapse \\ \hline
        \gls{TMD}~\cite{yang2022high} & High temperature/voltage & Dissolution of transition metals and \glspl{Li-ion} due to \gls{HF} \\ \hline
        \gls{CEI} film~\cite{jie2020advanced} & High voltage/\gls{SOC} & \glspl{Li-ion} depletion via electrolyte reactions \\ \hline
        Binder decomposition~\cite{zhao2020rechargeable} & High temperature & Loss of electrode contact \\ \hline
    \end{tabularx}
    \label{tab:cathode_aging}
\end{table}

\subsubsection{Electrolyte}

The electrolyte, crucial for \gls{LiB} stability, predominantly uses \gls{LiPF6}. 
However, \gls{LiPF6} decomposes into \gls{LiF} and \gls{PF5}, depending on carbonate types like \gls{EMC}, \gls{DMC}, and \gls{DEC}~\cite{YANG2006573}. 
Figure~\ref{fig:electro_reactions} outlines the primary degradation pathways. 
High voltage and \gls{SOC} conditions accelerate \gls{HF} formation, corroding the cathode and producing degradation byproducts. 
Elevated temperatures exacerbate breakdown, generating gases and further impairing performance~\cite{Lu2013}.

\begin{figure}[hbt]
    \centering
    \includegraphics[width=\textwidth]{Chapter2/figures/Electro_reactions2.jpg}
    \caption{Electrolyte degradation under high voltage and \gls{SOC}.}
    \label{fig:electro_reactions}
\end{figure}

\subsection{Battery Degradation in Battery Electric Vehicles}

The \gls{BMS} plays a crucial role in mitigating \gls{LiB} degradation by controlling variables such as \gls{SOC}, \gls{SOH}, thermal management, control units, and cell monitoring~\cite{Ramkumar2022}. 
Advanced \glspl{BMS} have adopted predictive techniques for \gls{SOC} and \gls{SOH} using methods like \gls{KF}, \gls{ANN}, and \gls{SVM}~\cite{Liu2022}. 
During charging, \gls{BMS} regulates voltage, current, and temperature with adaptive modes to extend battery life. 
However, unforeseen conditions inevitably contribute to battery aging.

Variables beyond \gls{BMS} control, such as \gls{SOC}, \gls{DOD}, and current rate, significantly influence battery life. 
High \gls{DOD} dramatically reduces lifespan, as shown by Han et al.~\cite{han2019review}. 
While small \glspl{DOD} are ideal for longevity, real-world requirements often necessitate larger \glspl{DOD}. 
Although \glspl{BMS} employ \gls{SOP} and \gls{SOE} estimators to forecast energy and power needs within 10–30 seconds, current rate depends on driver behavior rather than \gls{BMS} optimization. 
Guo et al.~\cite{guo2021lithium} emphasize the need for research on \gls{DOD} and current profiles under practical conditions. These uncontrollable factors are central to this study.

Standby Aging, or calendar aging, significantly affects battery life, accounting for up to 75\% of battery degradation as idle time constitutes 90\% of most \glspl{BEV}'s lifespan~\cite{EDDAHECH20122438,7045522}. 
Stroe and Schaltz~\cite{8911243} found that idling at 50\% \gls{SOC} accelerates aging more than at 10\% or 90\% \gls{SOC} at 25°C. Storage above 70\% \gls{SOC} further exacerbates fading at 45°C~\cite{BANK2020228566}. 
Elevated \gls{SOC}, voltage, and temperature during idle conditions accelerate degradation. 
Ideally, low \gls{SOC} and temperatures should be maintained, but achieving this in real-world scenarios remains challenging. 
Standby aging assessment is further complicated as \glspl{BEV}, in off-state, conserve energy by disabling monitoring systems.

Thermal management remains a critical challenge due to complexities in battery pack design and heat dissipation~\cite{wang2016critical}. 
Cell wiring configurations, such as series and parallel connections, influence temperature distribution. 
Uneven heat generation among cells complicates monitoring, with temperature sensors impacted by thermal delay effects~\cite{TALELE2021103482}. 
Core heat must propagate outward, creating spatial and temporal temperature gradients~\cite{wang2020study,chiu2014cycle}. 
These challenges hinder precise thermal regulation and contribute to degradation.

Driving behavior is a key factor in battery degradation~\cite{donkers2020influence}. 
Parameters such as current load, temperature, and voltage significantly impact \gls{LiB} lifespan~\cite{tremblay2009experimental,trippe2014charging,keil2016calendar}. 
Aggressive driving, characterized by rapid acceleration and deceleration (3–4 m/$s^2$), increases current load and often exceeds voltage thresholds, accelerating degradation~\cite{wang2018battery}. 
Elevated temperatures and mechanical stress further exacerbate wear~\cite{han2019review}. Studies have identified aggressive and negligent driving as detrimental practices~\cite{Alkinani2020}.

To address these challenges, advanced detection techniques developed for autonomous vehicles, safety, and fuel efficiency can be adapted for battery monitoring~\cite{Manzoni2010,Wang2022,autodrive2021}. 
These techniques utilize \gls{IMU}, \gls{GPS} trajectories, radar, cameras, and \gls{OBDII} data~\cite{Monselise2022,Zhy2022,WangJ2022,Zhu2022}. 
Statistical models (\gls{HMM}, \gls{GMM}, \gls{SVM}, etc.) and time-series deep learning methods (\gls{DNN}, \gls{RNN}, \gls{CNN}) offer robust tools for understanding driving behavior~\cite{deng2021review}. 
methods excel in efficiency but require expert feature extraction, whereas deep learning models deliver high performance at the cost of computational demands. 
Subsequent sections delve into these techniques and their applications for battery analysis.


\section{Control Recognition in Battery Electric Vehicles}

\subsection{Conventional Driving Behavior Recognition}

This section provides an overview of the common methodologies used to identify driving behavior, which is regarded as the human control input.
Statistical, mathematical, and empirical analyses of driving data form the foundation for these models. 
Methods based on CAN bus data include \gls{HMM}, \gls{GMM}, \gls{SVM}, \gls{NB}, \gls{FL}, and \gls{KNN}.

\subsubsection{Hidden Markov Model}

For low-level or fragmented driving behaviors, \gls{HMM} and its derivatives provide a simple solution. 
Using CAN data, \gls{HMM} can predict driving intents such as turns, stops, and lane changes with up to 90\% accuracy within two seconds~\cite{tran2015hidden}. 
Enhanced \gls{HMM} models, such as hierarchical \gls{HMM} (\gls{HHMM}), describe multi-layered connections to recognize high-level actions~\cite{fine1998hierarchical, he2012driving}. 
Derived models calculate results by aggregating outputs from low-level operations~\cite{zhu2016driving}. 
However, \gls{HMM} requires embedding to accommodate sequential data. The \gls{AR-HMM}, proposed by Abe et al.~\cite{abe2007study}, 
incorporates temporal order into sample processing, improving outcomes. Feature extraction, model structure, and segmentation continue to influence \gls{HMM} performance.

\subsubsection{Gaussian Mixture Model}

The \gls{GMM} is a probabilistic model that maximizes data log-likelihood across Gaussian distributions, handling overlapping clusters and non-circular shapes. 
Using Expectation Maximization, \gls{GMM} assigns data points probabilistically. 
In behavior analysis, \gls{GMM} identifies low-level driving behaviors. 
Miyajima et al.~\cite{Miyajima2007} modeled pedal operation patterns, while Carmona et al.~\cite{Carmona2016} demonstrated clustering aggressive and non-aggressive motions with 89.21\% and 92.65\% accuracy, respectively. 
Mardi et al.~\cite{Mardi2021} integrated \gls{GMM} with preprocessing and \gls{SVM}, achieving higher accuracy in smartwatch-based behavior detection. 
Furthermore, \gls{GMM} can label data for other models, enhancing driver risk evaluation~\cite{Song2021}.

\subsubsection{Support Vector Machine}

\gls{SVM} addresses classification issues for isolated or fragmented driving behaviors~\cite{Li2017}. 
Wang et al.~\cite{Wang2020} estimated time-to-collision using spatiotemporal inputs, achieving 80\% accuracy with data from CAN bus, radar, GPS, and video. 
Amsalu et al.~\cite{Amsalu2015} hybridized \gls{SVM} with \gls{HMM} and \gls{HSS}, achieving 97\% accuracy in intersection driving behavior estimation. 
\gls{S3VM} improves performance by leveraging semi-supervised clustering, requiring optimal cluster determination via methods like the elbow approach~\cite{Yang2021, Wang2017}. 
While \gls{SVM} excels in low-level motion analysis, its effectiveness depends on input feature diversity and quality.

\subsubsection{Naive Bayes}

The \gls{NB} model identifies driving behavior using conditional probabilities. 
It categorizes actions (low, medium, or high) by summarizing low-dimensional variables~\cite{Bouslimi2005}. 
Bayesian models effectively group driving behaviors, as demonstrated with accelerometer data~\cite{chen2015d}. 
Wu et al.~\cite{Wu2018} highlighted the benefits of visualizing features in 3D for better classification. 
However, \gls{NB} relies on extensive manual feature extraction and diverse datasets for robust analysis.

\subsubsection{Fuzzy Logic}

\gls{FL} models vague classifications through if-then rules, enabling precise behavior prediction. 
Fernández and Ito~\cite{Fernandez2016} categorized aggressive and passive driving styles using pedal frequency and speed. 
Deng and Söffke~\cite{deng2018improved} integrated \gls{FL} with \gls{HMM}, improving behavior prediction in complex scenarios. 
Li et al.~\cite{Li2022} developed a fuzzy-macro \gls{LSTM}, computing unsafe levels from abrupt actions and average speeds. 
While effective, \gls{FL} depends on expert-defined feature thresholds.

\subsubsection{K-Nearest Neighbor}

\gls{KNN} is a simple, non-parametric method for regression and classification. 
Li et al.~\cite{Li2017} used \gls{KNN} to identify driving behaviors, achieving accuracy within 2–6\% of \gls{SVM} and \gls{ANN}. 
Karri et al.~\cite{Karri2021} classified behaviors at intersections using throttle and braking pedal data, achieving over 90\% accuracy. 
High computational cost and parameter sensitivity (e.g., optimal \( K \)) remain challenges.

\subsection{Deep Learning Driving Behavior Recognition}

Deep learning offers high accuracy in driving behavior identification by leveraging time-series data from the CAN bus. 
Preprocessing steps such as segmentation, normalization, and filtering are critical for effective model input preparation.

\subsubsection{Recurrent Neural Network}

\gls{RNN}, including its variants like \gls{LSTM}, captures temporal dependencies in data. 
Jiang et al.~\cite{Jiang2021} predicted battery temperature with minimal error using \gls{LSTM} and \gls{GRU}. 
Zeng et al.~\cite{Zeng2021} used \gls{LSTM} to measure ride comfort and provide real-time driving assistance. 
Proper preprocessing and input formulation significantly impact \gls{RNN} accuracy~\cite{petnehazi2019recurrent}.

\subsubsection{Convolutional Neural Network}

\gls{CNN} excels in feature extraction and spatiotemporal analysis. 
Azadani et al.~\cite{azadani2020performance} demonstrated \gls{CNN} accuracy exceeding 96\% in driving behavior recognition. 
Shahverdy et al.~\cite{Shahverdy2020} transformed driving signals into images for \gls{CNN} input, achieving nearly 100\% accuracy. 
Heatmaps and class activation maps provide insights into feature representations~\cite{Selvaraju2016}.

\subsubsection{Fusion Models}

Fusion models integrate \gls{CNN} and \gls{RNN} architectures to enhance spatial-temporal activity recognition. 
Zhang et al.~\cite{Zhang2023} achieved 98.99\% accuracy with a \gls{CNN}-\gls{BiGRU} fusion model. Attention mechanisms further refine model performance, as shown by Zhang et al.~\cite{Zhang2019attention}.


\subsection{Battery Data Collection and Analysis}

Understanding the interplay between \glspl{LiB} and \glspl{EV} is pivotal for improving performance and reliability. 
Data-driven approaches show immense potential in advancing this comprehension. 
Two primary methods are commonly employed to collect battery-related data for analysis: 
(1) extracting data via the \gls{CAN} bus, and (2) performing terminal charge and discharge tests on the battery pack, 
which requires the battery pack to be completely removed from the \gls{BEV}. 
Given the logistical challenges of battery pack removal, this research adopts data collection through the \gls{CAN} bus as a more practical and feasible method.

Gonçalo et al.~\cite{dosreisG2021} provided a comprehensive categorization of commonly used data types in battery management system (\gls{BMS}) research, 
including laboratory data, real-world driving cycle data, and synthetic data. 
Laboratory data, such as \gls{SOC} and \gls{SOH}, serve as the theoretical foundation for various measurements and analyses. 
Synthetic data, on the other hand, is often employed to supplement existing datasets and simulate extreme conditions, enabling robust evaluation of the \gls{BMS}. 
Real-world driving cycle data are indispensable for validating performance against national standards and ensuring the certification of \glspl{EV}. 
However, while these datasets provide valuable insights, most focus predominantly on battery-specific metrics, with limited emphasis on actual driving behaviors and activities.

Real-world data play a crucial role in addressing additional challenges in \gls{EV} operations, such as fault diagnostics, thermal management, and health management~\cite{waldmann2014temperature}. 
These challenges include estimating battery aging, remaining useful life (\gls{RUL}), end of life (\gls{EOL}), and distance traveled. 
Compared to lab-generated data and simulations, real-world \gls{EV} data reflect unique operational and environmental factors that lead to distinct degradation patterns. 
Driving cycle data—such as the \gls{WLTC} and \gls{NEDC}—further demonstrate, 
through comprehensive natural-world investigations, that variations in operational conditions and environments significantly impact battery performance~\cite{safdari2022numerical, 9424412}. 
Consequently, accurate assessments of battery health and aging mechanisms require real-world data rather than relying solely on standardized laboratory criteria.

Despite the value of real-world data, accessing it remains a significant challenge in the industry. 
Manufacturers are often reluctant to share detailed information, and the process of collecting data via the \gls{OBDII} interface is intricate. 
Proprietary protocols are employed to safeguard vehicles and prevent competitors from exploiting the data, making it virtually impossible to use off-the-shelf \gls{IoT} devices for data extraction. 
Accessing \gls{OBDII} data typically necessitates the development of customized \gls{IoT} solutions tailored to the specific protocols used by manufacturers.

To facilitate comparisons between our collected data and other publicly available datasets, Table~\ref{tab:public_data} compiles a list of potential data sources. 
This includes \gls{EV} \gls{CAN} data summarized by Gonçalo et al.~\cite{dosreisG2021}, expanding the scope of the analysis.

\begin{table}[!ht]
    \centering
    \caption{The Public \gls{CAN} bus data for driving behavior recognition}
    \begin{tabularx}{\textwidth}{ccccXX}
    \hline
        Dataset & Access & No. of features & Sampling rate& Battery info. & Labeled\\ 
        \hline
        \cite{kwak2016know} & \cite{hcrl} & 54 & 1s & - & Driver \\
        \cite{oh2020vehicle} & \cite{oh_2023_ved} & 22 & 1s & SOC, A, V & Trip \\
        \cite{6jr9-5235-20} & \cite{kagglebattery} & 28 & 0.1s & A, V, C & - \\
        \hline
    \end{tabularx}
    \begin{tablenotes}
      \small
      \item - A: \gls{HV} battery current; V: voltage; C: temperature;
    \end{tablenotes}
    \label{tab:public_data}
\end{table}



\section{Reinforcement Learning in Multi-Objective Control} %Reinforcement Learning in Autonomous Control: A Multi-Objective Perspective

Review RL applications in control systems, with a focus on balancing multiple objectives (e.g., performance and energy efficiency).

Discuss the challenges of integrating RL with real-world constraints like SOH predictions and perception inputs.

Summarize RL applications in autonomous systems.
Highlight multi-objective RL approaches for balancing performance and energy constraints.


\section{Generalization and Simulation for Autonomous Systems}

Explore strategies like domain randomization, synthetic data generation, and few-shot learning for improving model robustness.


Highlight your work on Blender-generated datasets and Grounding DINO for efficient training with minimal data.


\section{Ethical, Societal, and Regulatory Considerations}


Address ethical issues, such as data privacy in perception systems and safety implications of SOH-aware control.

Discuss relevant regulations for deploying autonomous systems in public domains.


\endinput